% introducao
%
\begin{introducao}
    \chapter{Introdução}
    \label{chap:introducao}


O tema foi escolhido devido à crescente importância das tecnologias de visão computacional na área médica,
especialmente no que diz respeito a exames de imagem. A mamografia é um exame de imagem
que utiliza raios-X para detectar alterações suspeitas nas mamas, e é o principal método de rastreamento
para o câncer de mama. A detecção precoce do câncer de mama é fundamental para o sucesso do tratamento, e a 
mamografia é o método mais eficaz para esse fim. No entanto, a interpretação de mamografias é um processo 
complexo e sujeito a erros, e a taxa de falsos positivos e falsos negativos é alta.
Assim, o desenvolvimento de métodos de auxílio ao diagnóstico é de grande importância para a área médica.

O objetivo deste trabalho é avaliar o desempenho de redes neurais convolucionais na classificação de imagens
médicas, utilizando como base de dados o \textit{dataset} \textit{CBIS-DDSM} \cite{cbis-ddsm}.

\section{Motivação}
\label{sec:motivacao}

O câncer de mama é o segundo tipo de câncer mais comum no mundo, e o mais comum entre as mulheres.
Em 2018, foram estimados 2,1 milhões de novos casos de câncer de mama, e 627 mil mortes causadas pela doença \cite{cancer-stats}.

\end{introducao}


